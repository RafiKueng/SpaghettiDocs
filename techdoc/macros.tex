
\newcommand{\spl}{SpaghettiLens\xspace}
\newcommand{\sw}{SpaceWarps\xspace}

% shortcut for einstein radius (Text Greek Full Math)
\newcommand{\ER}{Einstein radius\xspace} % text
\newcommand{\ERg}[1][]{$\Theta_\text{E#1}$\xspace} % er in textmode with greek
\newcommand{\ERf}[1][]{Einstein radius $\Theta_\text{E#1}$\xspace} % full output
\newcommand{\ERm}[1][]{\Theta_\text{E#1}} % mathmode with greek symbol

%shortcuts for kappa
\newcommand{\kenc}[1][r]{$\kappa_\text{encl}(#1)$\xspace}
\newcommand{\kap}[1][r]{$\kappa(#1)$\xspace}

% shorcuts for refs (use capital for beginning of sentence)
% first 3 are for the real layz people..
\newcommand{\fref}[1]{\ref{fig:#1}}
\newcommand{\sref}[1]{\ref{sec:#1}}
\newcommand{\tref}[1]{\ref{tab:#1}}
\newcommand{\lref}[1]{\ref{lst:#1}}
\newcommand{\uref}[1]{\ref{uml:#1}}
\newcommand{\figref}[1]{Figure~\ref{fig:#1}}
\newcommand{\secref}[1]{Section~\ref{sec:#1}}
\newcommand{\tabref}[1]{Table~\ref{tab:#1}}
\newcommand{\lstref}[1]{Listings~\ref{lst:#1}}
\newcommand{\umlref}[1]{UML~Diagram~\ref{uml:#1}}
\newcommand{\Figref}[1]{Figure~\ref{fig:#1}}
\newcommand{\Secref}[1]{Section~\ref{sec:#1}}
\newcommand{\Tabref}[1]{Table~\ref{tab:#1}}
\newcommand{\Lstref}[1]{Listing~\ref{lst:#1}}
\newcommand{\Umlref}[1]{UML~Diagram~\ref{uml:#1}}

% reference to ranges of listings (for multipage..)
% \lstrefr[n_pages]{base filename}
\newcommand{\lstrefr}[2][1]{\crefrange{lst:#2_1}{lst:#2_#1}}
\newcommand{\Lstrefr}[2][1]{\Crefrange{lst:#2_1}{lst:#2_#1}}


% shortcut for ASW000xxxx
\newcommand{\asw}[1]{ASW000#1\xspace}

%shorcut for model (maybe link later to appendix)
% use \model{4356} for with text and
% \model[]{3456} for only number
\newcommand{\model}[2][Model~]{#1#2\xspace}

\newcommand{\code}[2]{%
\begin{listing}%
  \centering%
  \input{code/#1}%
  \caption{#2}%
  \label{lst:#1}%
\end{listing}%
}



% insert multiple pages of code
% \codep[npages]{base file name}{caption}
\newcommand{\codep}[3][1]{%
  \foreach \index in {1, ..., #1} {%
    \code{#2_\index}{#3 (page \index\xspace of #1)}
  }%
}

\newcommand{\uml}[3][]{%
\begin{figure}%
  \centering%
  \includegraphics[#1]{uml/#2.pdf}%
  \caption{#3}%
  \label{uml:#2}%
\end{figure}%
}

% small code fragments

% class / object name
\newcommand{\C}[1]{%
\path{#1}%
}
%method
\newcommand{\M}[1]{%
\path{#1}%
}
% function
\newcommand{\F}[1]{%
\path{#1}%
}
% some other code
\newcommand{\T}[1]{%
\path{#1}%
}
% strings
\newcommand{\str}[1]{%
{\ttfamily "#1"}%
}
% events
\newcommand{\evt}[2][]{%
\path{#1}%
}

% interface
\newcommand{\I}[1]{{\ttfamily/#1/}}


% code names
\newcommand{\lmt}[1]{\C{LMT.#1}\xspace}
\newcommand{\lmto}[1]{\C{LMT.objects.#1}\xspace}

% file names
\newcommand{\fjs}[1]{{\ttfamily lmt.#1.js}\xspace}

\DeclareUrlCommand\path{\urlstyle{tt}}

% url to spaghettilens
\newcommand{\splurl}[1][]{\url{http://spaghettilens.url/#1}}


% backend stuff

%backend filename
\newcommand{\befn}[1]{{\ttfamily /backend/#1}\xspace}






%%% STANDART LAENGENANGABEN %%%%%%%%%%%%%%%%%%%%%%%%%%%%%%%%%%%%%%%%%%%%%%%%%%%
\newlength{\figwidth}
\setlength{\figwidth}{0.8\textwidth}


