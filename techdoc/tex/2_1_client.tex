\subsection{Client}
\label{sec:client}

In webapp programming, is basically an interactive webpage, that communicates with other components.

To accive that, you need 3 technologies: html, css and javascript. This is a unique standart.

HTML describes the logical structre and layout of your webpage, the userinterface. That elements are on the page.

CSS defines the visuals, optics. how those elements defined by HTML are displayed. How those elements on the page look like

Javascript (js) is the programming language, that runs on the client computer and thus defines the behaviour. It describes, how the elemnts interact with each other.


\spl was designed to do as much computation as possible on the client side. This removes the load on the application server, network traffic and leads to a more natural feeling. The app responds instantaniouly to user inputs (instead of having to wait for a server response). Drawback: you have less control over data (saving snapshots) since browser are usually encapsulated from the rest of the computer, difficult to save status / data client side.


An interactive webpage / user interface needs javascript for document object model (DOM, the tree of html attributes describing the structre of the webpage) manupulation.
\spl consists of a single website.
In opposite to most classical websites, where you click a link and load another page to react to user input, \spl is only one single page, that updates it's strucrte live as the user interacts with the page (called DHTML, dynamic HTML).
Thus a dhtml webpage resembles more a classical desktop application than a classical website.

If your DHTML site needs to exchange information with a server (e.g. to do some heavy calculations, store and retrieve data from a central database) you get Ajax (Asynchronous JavaScript and XML).




ECMAScript (aka JavaScript)


The modern webapp has thus 4 components: user input, output, intelligence, communications.
All of which is programmed in javascript.
Since this is a very common task, there are several libraries implemented in javascript, that facilitate the DOM manipulation (and thus creating a user interface).

For \spl the the most common one was chosen, jQuery.
It is used by two thirds of the top 10'000 webpages.
It is open source and due to it's widespread use, it guarantees to be kept up to date and get often bugfixes.

jQuery offers several addons, one of them jQuery UI, that facilitates the einheitliches design of user interfaces

basically, jquery makes it easy to find, access and modify elements in the DOM, as shows \lstref{jquery_vs_js.js}.

\code{jquery_vs_js.js}{Comparing the modification of some objects, done in jQuery and pure js.}




HTML

There are two versions of html: the actual version 4.01, which almost anything that is connected to the internet is compatible to, probably even you toaster. Version 4 is around since Dec. 1998.

And the next generation HTML5, that is currently in developpent (First working draft from Jan 2008).
It offers many new features, for audio, video and graphics (canvas), communications (websockets), local storage.
But is only supported by newer Web Browser and thus a smaller user base.

Never the less we decided to go for html5. Because if offers the canvas and svg, two essential techniques explaned later for a web app.
Since our expected user base will origin from galaxzoo and \sw, and those sites are also programmed in HTML5, we could expect our users to have up to date browsers.





The visual design:



