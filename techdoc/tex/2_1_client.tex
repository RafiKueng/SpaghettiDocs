\subsection{Client}
\label{sec:client}

\subsubsection{Overview}
\label{sec:client_overview}


\subsubsection{Techniques}
\label{sec:client_techniques}



In webapp programming, is basically an interactive webpage, that communicates with other components.

To accive that, you need 3 technologies: html, css and javascript. This is a unique standart.

HTML describes the logical structre and layout of your webpage, the userinterface. That elements are on the page.

CSS defines the visuals, optics. how those elements defined by HTML are displayed. How those elements on the page look like

Javascript (js) is the programming language, that runs on the client computer and thus defines the behaviour. It describes, how the elemnts interact with each other.


\spl was designed to do as much computation as possible on the client side.
This removes the load on the application server, network traffic and leads to a more natural feeling.
The app responds instantaniouly to user inputs (instead of having to wait for a server response).
Drawback: you have less control over data (saving snapshots) since browser are usually encapsulated from the rest of the computer, difficult to save status / data client side.


An interactive webpage / user interface needs javascript for document object model (DOM, the tree of html attributes describing the structre of the webpage) manupulation.
\spl consists of a single website.
In opposite to most classical websites, where you click a link and load another page to react to user input, \spl is only one single page, that updates it's strucrte live as the user interacts with the page (called DHTML, dynamic HTML).
Thus a dhtml webpage resembles more a classical desktop application than a classical website.

If your DHTML site needs to exchange information with a server (e.g. to do some heavy calculations, store and retrieve data from a central database) you get Ajax (Asynchronous JavaScript and XML).




ECMAScript (aka JavaScript)


The modern webapp has thus 4 components: user input, output, intelligence, communications.
All of which is programmed in javascript.
Since this is a very common task, there are several libraries implemented in javascript, that facilitate the DOM manipulation (and thus creating a user interface).

For \spl the the most common one was chosen, jQuery.
It is used by two thirds of the top 10'000 webpages.
It is open source and due to it's widespread use, it guarantees to be kept up to date and get often bugfixes.

jQuery offers several addons, one of them jQuery UI, that facilitates the einheitliches design of user interfaces

basically, jquery makes it easy to find, access and modify elements in the DOM, as shows \lstref{jquery_vs_js.js}.

\code{jquery_vs_js.js}{Comparing the modification of some objects, done in jQuery and pure js.}




HTML

There are two versions of html: the actual version 4.01, which almost anything that is connected to the internet is compatible to, probably even you toaster. Version 4 is around since Dec. 1998.

And the next generation HTML5, that is currently in developpent (First working draft from Jan 2008).
It offers many new features, for audio, video and graphics (canvas), communications (websockets), local storage.
But is only supported by newer Web Browser and thus a smaller user base.

Never the less we decided to go for html5. Because if offers the canvas and svg, two essential techniques explaned later for a web app.
Since our expected user base will origin from galaxzoo and \sw, and those sites are also programmed in HTML5, we could expect our users to have up to date browsers.



\subsubsection{Visual Layout}
\label{sec:client_vis_layout}

The User Interface was designed to be as simple as possible.
It should hide away all settings that are not needed by default.
To be self explanatory is almost impossible given the problem.
Users will need to see a tutorial anyways.

The basic degin idea was to provide a visual feedback and easy comparison between the input modelling parameter and the resulting output model.
This is accived by putting the input area and output area side by side.
All the functions that the user needs to manupulate the input are arranged on top of it, everything manipulating the output above the output window.
A few general commands are arranged in a top bar.

To assist the user, two systems are present: A exhaustive mouse over hoover tooltip help that pops up any relevant information for an tool / button under the cursor.
This provides a short description of the action, and possibly a link to athe tutorial page.

Additionally, there is a help bar at the bottom. This can be hidden from the toolbar.
It also provides information about objects in the input area.
mouse over tooltips would not function in the input area well, obstructing the users view over the model, hence this two folded helping system.

For clients with a small screen, like mobile devices and possibly tables, the layout changes to only display the input or the output. The user can slide between those to sides using a slider.\footnote{Note that mobiles weren't tested and are not officially supported yet.}


\subsubsection{Programm Layout}
\label{sec:client_prog_layout}

The client side application was programmed with a strict object oriented and event driven perspective.
Instead of a regular program that has a main loop, that runs infinitly and where subfunctions are called, a event driven program is at halt when idle, and reacts to event that can be fired by any source (example: a key press triggers a function, instead of looping infintely and checking if any key was pressed.)
This is a very simplified abstractaion\footnote{and technically not entierly true, there is an main loop that checks if events happend and then a dispacher calls the registered function.} but describes the situation of running javascript in a browser quite well.

Everything is a object. The toolbar, the model, a dialog screen.
All objects can fire events and react to events and thus change their internal state, and/or fire a subsequent event.
There are unique objects (like toolbar, input area) and instances of prototype objects\footnote{in other programming languages prototype objects are called classes}. There can be more than one instance of a prototype object (for example Extremal Points, Point masses...)

The program is always in a certain state, defined throu the attributes of all the objects. In \spl, the state consists of the state of the model, and the state of the ui.

The main part of \spl is the object lmt.events. This object registers at startup, which objects react with witch object function on what events and thus, this object defines all events possible (except a few standard events).

There are basically three groups of objects that interact with each other: The user interface objects (LMT.ui, split in LMT.ui.svg, LMT.ui.out, LMT.ui.html), the comunications object (LMT.com) and the current model (LMT.model, instance of LMT.objects.model)

A model consists of an array of sources, an array of external masses (instances of LMT.objects.ExternalMass) and an parameter object. Each source is represented by an tree structure of instances of extremalpoints (LMT.objects.ExtremalPoint)

Additionally, there is an action stack (LMT.actionstack), LMT.datasource, LMT.datasources

and some objects that only store common data / functions that needs to be accessed by multiple objects (LMT.modelData, LMT.settings, LMT.simulationResult, LMT.utils)



