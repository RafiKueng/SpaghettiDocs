\section{Introduction}

Gravitational lenses (GLs) are a great tool to be used by astronomers to study properties of the cosmos.
But before they can be used for further studies, they have to be spotted.

This is a hard task, given the huge amount of data observatories produce and the average performance of existing robotic lens detection tools.
\sw\footnote{\protect\url{http://www.spacewarps.org}} addresses this problem by inviting volunteers to help spot GLs, with a great success.
Over 1.7 million images have been classified in the first two weeks after the start of the citizen science project.

The next step is to analyze and model the lens candidates found.
If a convincing model can be created for a lens candidate, that supports the hypothesis, that it is really a lens. Further, it is the first step in a more detailed analysis of the lens.

This is a rather sophisticated, time intensive task usually done by scientists.
It was doubted that the process of  modeling a gravitational lens can be reliably done by volunteers.

We proposed that it is possible, if you present the underlying theory in a compact, easy to understand way and provide an easy to use tool that gives the volunteers a visual feedback of the modeling process.
Such a tool could also be useful for scientists to quickly produce models of GLs.

This report introduces the tool written for this task: \spl. It was part of my Masters Thesis and details the design and set up of the tool. It's intention is to be a technical documentation for any body that wants to understand and contribute to the application.

The scientific part of my Masters Thesis was to test our proposal by letting volunteers model simulated lenses. The known parameters of the simulations were compared to the parameters of the models generated by the volunteers. The results of this part can be found in the scientific paper (to be published, you can have a look at the draft at \url{http://www.physik.uzh.ch/~rafik/?f=slp.pdf})

Additional parts included the tutorial video, explaining the theory and use of \spl\footnote{\protect\url{http://mite.physik.uzh.ch/tutorial/}} and a few web sites that will soon come online.


%, a tutorial homepage giving written explanation of whats shown in the video and a sub page for \sw called labs, where different modelng tools will be collected and presented. The last two are not yet online.

\subsection{\spl}

\spl is build on top of GLASS\cite{glass-jc} that implements non parametric, pixel based lens modeling.
GLASS is a reimplemented version of PixeLens\cite{pixelens}.
%
\spl provides a easy to use web based user interface offering:
\begin{itemize}
  \item Quickly create models of lenses from different data sources (\sw, \ml \footnote{\protect\url{www.masterlens.org}})
  \item Help volunteers to collectivity work on generating models
  \item Help scientists store, manage and evaluate models and data for lenses
\end{itemize}

It was designed using modern web technologies to implement a powerful, modular web application.
Chapter \ref{sec:primer} gives an introduction to current web technologies applied in this project.
The main part of this report is Chapter~\ref{sec:setup}, explaining all the parts \spl is made of in detail, whereas Chapter~\ref{sec:pd_flow} details the interplay between the modules.


