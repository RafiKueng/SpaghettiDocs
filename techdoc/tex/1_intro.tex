\section{Introduction}

Gravitational lenses (GLs) are a great tool for astronomers to study properties of the cosmos.
But before they can be used for further studies, they have to be spotted.

This is a hard task, given the huge amount of data which observatories produce and the average performance of existing robotic lens detection tools.
\sw\footnote{\protect\url{http://www.spacewarps.org}} addresses this problem by inviting volunteers to help spot GLs, with a great success.
Over 1.7 million images have been classified in the first two weeks after the start of the citizen science project.

The next step is to analyze and model the lens candidates found.
A convincing model for a lens candidate supports the hypothesis that it is really a lens. Further, it is the first step in a more detailed analysis of the lens.
It allows for example the analysis of the matter distribution including dark matter.

Creating a model is a rather sophisticated, time intensive task usually done by scientists.
We proposed that it is possible for non scientists (volunteers\footnote{also known as citizen scientists}) to create models that are comparable to those of scientist. This requires in a first step to present the underlying theory in a compact and simple way.
Second, a modeling tool has to be developed that simplifies the process of creating a model.
It would be preferable, if this tool provides an instant, meaningful visual output.
That would allow the volunteers to improve their models iteratively, by try and error.
In a third step, the building of a community of users should be encouraged.
On one side by closely working together to further adjust the tool to the users needs, on the other side to assist if any questions come up.


This report introduces the tool written for this task: \spl.
It was part of the authors Master Thesis and details the design and setup of the tool.
This is intended to be a technical documentation for anybody who wants to understand and contribute to the application.

The scientific part of my Master Thesis is to test our proposal by letting volunteers model simulated lenses.
The known parameters of the simulations were compared to the parameters of the models generated by the volunteers.
The results of this part can be found in the scientific paper (to be published, you can have a look at the draft at \url{http://www.physik.uzh.ch/~rafik/?f=slp.pdf})

Additionally, a tutorial video was recorded, which explains the theory and use of \spl\footnote{\protect\url{http://mite.physik.uzh.ch/tutorial/}}.


%, a tutorial homepage giving written explanation of whats shown in the video and a sub page for \sw called labs, where different modelng tools will be collected and presented. The last two are not yet online.

\subsection{\spl}

\spl is build around GLASS\cite{glass-jc}, a non parametric, pixel based lens modeling tool set.
GLASS is a reimplemented version of PixeLens\cite{pixelens}
%
\spl is build as a client-server web application that runs on any computer in the browser without any installation\footnote{Besides an up to date browser}.
If offers the following features:
\begin{itemize}
  \item Quickly create models of lenses, originating from different data sources (\sw, \ml \footnote{\protect\url{http://www.masterlens.org}})
  \item Users can easily present their results, discuss and improve them
  \item Allows scientists to store, manage and evaluate models and data for lenses
\end{itemize}

It was designed using modern web technologies to implement a powerful, modular web application.
Chapter \ref{sec:primer} gives an introduction to current web technologies applied in this project.
The main part of this report is Chapter~\ref{sec:setup}, explaining all the parts \spl is made of in detail, whereas Chapter~\ref{sec:pd_flow} details the interplay between the modules.
Chapter~\ref{sec:deployment} gives a short overview over the installation and maintenance and finally some review and outlook in Chapter~\ref{sec:outlook}.


