\subsection{Proxy}
\label{sec:proxy}

In the current setup, there is a proxy server between the application server and the internet.
It's purpose is to reduce the load of the application server and to prevent the firing up of a resource intensive\footnote{CPU time and memory size} Python thread in the app server on every request.

It is supposed to directly serve the client application. Further, this particular server is set up to serve \url{labs.spacewarps.com} and the \spl documentation and tutorial page in combination with a php process.\footnote{this part is still work in progress}

It is setup to further intercept requests to \I{data} and to return the data directly, if available.
Only if the result needs to be rendered first, it is passed on to the application server.

It is also set up to cache any requests made to \I{api} that can be regarded as static, like getting the information about a lens.

At the moment, the proxy uses the web server software nginX (``Engine X'') that is preferred over Apache because of the small memory finger print.
This is due to the fact that nginX is event based, in opposite to Apache, which is process based.
nginX is often used as proxy server and load balancer to serve static content in front of another server like Apache that handles the dynamic content.



