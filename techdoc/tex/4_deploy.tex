\section{Configuration and Deployment}
\label{sec:deployment}

This section gives an overview of the possible setups and configuration of \spl.

\subsection{Configuration Files}

All relevant config files can be found in the \path{./backend/settings} directory.

\begin{itemize}
  \fbul{base_settings.py}{This file contains the basic configuration of Django and internal Django setup. This should stay the same for all possible configurations.}
  \fbul{gunicorn.py}{This is the configuration file for the WSGI server running Django. Only used for the full server setup.}
  \fbul{machine.py}{This file sets up machine specific paths, the database connection and URLS. The \T{ROLE} parameter defines what kind of installation this setup is and which modules should be used.}
  \fbul{secrets.py}{This file contains information that should be kept secret like user credentials for the database.}
  \fbul{settings.py}{This is the master settings file that import all others. There are no parts to be changed by the user. It defines all possible roles.}
  \fbul{version.py}{Keeps track of version numbers. Is automatically updated by the install scripts.}
\end{itemize}


\subsection{Full Server -- Installation}
\label{sec:serverinstall}

The full server installation installs all components on one machine that is supposed to serve as web server.
Worker threads can additionally be run on different machines, see \secref{workerinstall}.

Due to the complicated setup, this process is automated with a interactive installation script.
The installation script is tested on a fresh install of Ubuntu server edition, but should work with all Debian based systems.
It needs a lot of packages, modules and configuration files to be setup and will create and change some system configuration files.
A full install on a fresh installed OS will need about 1GB\footnote{Due to the many packages that need to be installed. Most systems already have most available.}.

Warning: The install script will write and change configuration files for MySQL, nginX and RabbitMQ.
If you have any of those running, please don't use the install script.

Requirements for the install scripts are a installed Python 2.7 with packet manager PiP and the packages numpy, scipy, matplotlib and fabric.
Additionally, you have to get a copy of GLASS.

The script is started by running the following command in the base directory:
\cmd{fab install}

All components should start up automatically on restart and \spl should be available.


\subsection{Full Server -- Update}

To update the server, first change to the source directory and make sure to have the latest version.
The update scripts will copy the new files to the according locations and minify the HTML, CSS and JS files into single files.
\cmd{fab update\_backend:install\_dir='/path/to/install'}
\cmd{fab update\_html:install\_dir='/path/to/install'}

If there were updates to the database design, the new schema will have to be manually applied.
\spl uses the Python module south to help with schema migrations.


\subsection{Worker Installation}
\label{sec:workerinstall}

Additional worker nodes can easily be spawned on any machine that has access to the file system of the server using SSH and can access the database.
No root access is needed on the local machine.
A step-by-step documentation of the installation is available in the file \file{./install/roles/production_worker.py}



\subsection{Other Installations}
If requested, the installation script can be modified to allow further installation types.
The installation script lets the user select, what type of install is desired.
Depending on the selection, it installs and configures the depended software.

This could easily be extended to support an installation on a local machine for private use.
Refer to the files in \path{./install/} for further information.

