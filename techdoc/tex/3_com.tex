\section{Program and Data Flow}
\label{sec:pd_flow}

This section gives an overview over the program and data flow over the major components for three non trivial cases using UML sequence diagrams.
Sequence diagrams show which components of a application are active at a given time.
They abstract internal processes in modules and depicts simplified function calls and return values.



\subsection{Selection of a Data Source}

The application start up involves two processes.
The user first has to choose a data source, followed by selecting a lens to work on.
This section shows the first process, until the display of the dialog to select a lens to work on.
The next section elaborates the continuation.

In a first step, all static content like HTML, CSS and JS files are loaded from the server.
This files are directly served by the proxy server.
Internally, the client application now starts the initialization routines, as soon as the JS files are loaded.

In a next step, the client application needs to know, what data sources are implemented on the server, to display a dialog to the user to select from.

The last step is to display the data source specific dialog, that allows the user to browse and select lenses, that are in the data sources data base.
The functionality of this dialog is implemented by the data source module selected and can vary among data sources.
The data source module generates the HTML markup used to generate the dialog in the client application.

\seq{init}{The two step initialization of \spl. First step loads the data selection dialog, followed by the data source specific lens selection dialog.}


\subsection{Selection of a Lens}

The second step at the start up of the application is the selection of a lens to work on.
This process differs depending on the selected data source in detail, but the overall idea is the same for all sources.
\seqref{dsinit} shows this process for the \sw module.
Note that the proxy server is not shown in this dialog, because it passes all the request on to the server.
Further, the server modules \befn{urls.py} and \befn{views.py} are combined, to simplify the diagram.

\seq{dsinit}{The selection and loading of a lens. First step (green) depicts the last action \seqref{init}; the selected id is validated in step two (yellow); validated lens data base entry is loaded or created if not available and the UI initialized (red).}


It involves to offer the user a selection for a lens.
This can be done for example using drop down fields or a simple text field.
The data source module has to guarantee that this selection is valid.
Otherwise, the user should not be able to continue.

The \sw data source offers a simple text filed for entering a \sw image id.
It then checks the \sw data base, whether this id exists.
If it does, it fetches additional information, displays it to the user and enables the button to continue.
Otherwise a error message is shown.
\seqref{dsinit} shows this part, colored in yellow.

The last step, colored in red in \seqref{dsinit}, involves three parts.
The data source module has to return a \T{model_id}, representing the selected lenses in the server side database.
It has to guarantee the uniqueness of each lens in the database.
It first needs to check if the selected lens(es) already exist in the database.
If it does, this id is returned. Otherwise a new entry is creating and returned.

In the second part, the main part of the client application now queries the data base for the lens to be shown next.
The response includes at least one URL to the actual image, that gets loaded in the third part and then rendered to the screen.

This concludes the start up procedure and hides all dialog pop ups, allowing the user to begin modeling.

Alternatively, the user could load \spl passing a GET argument \T{mid} or \T{rid}.
This start up skips all the parts involving the data source modules and directly queries the server data base for the lens information and loads it.
If a result is loaded using \T{rid}, an additional query to the data base is done, returning the JSON string representing the model.
After loading the lens data, the model data is loaded and shown.



\subsection{Simulation}

After the user created a model, it can be sent to the server to get simulated and once finished, retrieve the resulting figures. \seqref{simulate} shows the whole process.

\seq{simulate}{Asynchronous simulation of a model. Shows the client polling for results until they are ready (green); the task queue (yellow) and worker process executing a task (red).}

In a first step, the model JSON string is parsed, validated and evaluated.
It is saved to the data base as a new entry and the resulting id is returned to the client.
Additionally, a configuration file for the simulation back end is written to disk. 

Then the client asks for the location of the results of this model.
Since there are no results yet, the server creates a new simulation task and hands it over to the broker, that puts it in the task queue (\seqref{simulate}, yellow).
A reference to this task and the fact, that the simulation has started is saved in the data base and sent to the client.

As soon as a worker is idle, it consumes the next task in the queue and executes it (\seqref{simulate}, red).
A task consist of getting the configuration file, running GLASS, producing the output figures and uploading those back to the server file system.
The worker keeps track of the state of the task in the database.

Meanwhile, the client repeatedly checks with the server, whether the task has been completed.
If so, the server sends back links to the generated figures, that the client then loads and displays to the user.







