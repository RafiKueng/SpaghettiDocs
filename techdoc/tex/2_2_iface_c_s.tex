\subsection{Interface Client Server}
\label{sec:iface_c_s}

interface client server

%\codep[0]{json_repr_parts.js}{The building blocks of a model in JSON notation} 
%\codep[1]{json_repr_parts.js}{The building blocks of a model in JSON notation} 
%\codep[2]{json_repr_parts.js}{The building blocks of a model in JSON notation} 

%\codep{2}{json_repr_parts.js}{The building blocks of a model in JSON notation} 

%\begin{listing}%
  %\centering%
  %\begin{SubFloat}[bla]{\label{tmp1}Capt label1}
    %\input{code/tmp1}
  %\end{SubFloat}
  %
  %
  %
%%  \subfloat[\label{tmp1}]{\input{code/json_repr_parts.js_p0}}%
%%  \caption[\label{tmp1}]{}
  %%
    %%\label{lst:tmp1}%
%\end{listing}%
%
%
%\begin{listing}%
  %\ContinuedFloat
%
  %\begin{SubFloat}[bla2]{\label{tmp2}Capt label1}
    %\input{code/tmp2}
  %\end{SubFloat}
%
%
  %%\centering%
  %%\subfloat[\label{tmp2}]{\input{code/tmp1}}%
  %%\input{code/tmp2}%
  %%\caption[\label{tmp1}]{}
  %}%
  %%\label{lst:tmp2}%
%\end{listing}%



%\codep[0]{json_repr_parts.js}{The building blocks of a model in JSON notation} 
%\codep[1]{json_repr_parts.js}{The building blocks of a model in JSON notation} 



%\cref{lst:json_repr_parts.js_1,lst:json_repr_parts.js_2,lst:json_repr_parts.js_3}

The server offers three points to exchange data: \I{api}, \I{tools} and \I{data}.

Since there is some legacy code still in the code base, there are additional, depreciated url access points to exchange data. (\I{get\_initdata}, \I{get\_modeldata}, \I{save\_model}, \I{save\_model\_final}, \I{load\_model}, \I{result})

\I{api} is supposed to handle all communication with the client application.
The client can do a HTTP POST request to \splurl[api].
The requests body should contain at least the key \str{action} set to the value of the desired action.
The basic transmission format is JSON based.
Depending on the action chosen, additional key/value pairs need to be transmitted.
Have a look at the source code files \befn{ModellerApp/views.py} and/or \fjs{com} for details.



\codep[3]{json_repr_parts.js}{The building blocks of a model in JSON notation} 


\Lstrefr[3]{json_repr_parts.js} shows the JSON communication format for sending models between client and server in a structured notation. This has the same tree like structure as is shown in \umlref{model}.


\I{data} offers an access point, to present simulated models in an easy way.
It answers to a basic HTTP GET request by parsing the rest of the url to a integer result id and returning a simple web page showing this result number.
\splurl[<result-id>]\footnote{example: \splurl[1337]}


\I{tools} offers access for advanced users and admins to tools for administering collaborative modeling, getting more detailed data ect.
At the moment, only one tool is implemented: \I{tools/ResultDataTable}.
This tool creates an overview table over all parameters for a set of result ids.
The resulting table can be downloaded as a Excel file (\T{csv}) or directly shown in a browser (\T{HTML-table}).
The query to the data base is directly composed of the HTTP GET parameters:
\splurl[tools/ResultDataTable?6696,6904-7000\&type=html].
For a full documentation visit the tool without any argument\footnote{\splurl[tools/ResultDataTable]}.






