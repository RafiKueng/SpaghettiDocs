\section{Primer in Web Technologies}

This section gives a short overview over the techniques and standards used in web development.
It introduces the concept of 


\subsection{HyperText Markup Language (HTML)}
The HyperText Markup Language (HTML) is the markup language used as a format to exchange semantically structured information on the internet.
It is a textual language describing the building blocks of a web site.
Those building blocks are called tags.
They are enclosed in angle brackets and consist of opening tag, a set of attributes like id, class and additional depending on the tag type, and an closing tag, see \lstref{atag.html}

\code{atag.html}{Example HTML tag.}

This tags can enclose any content, data to be displayed or other tags. This results in an document tree, that can be accessed using the interface ``Document Object Model'' (DOM). Often, the document tree itself is called DOM\footnote{DOM describes actually an interface, not a model.}.



\subsection{Cascading Style Sheets (CSS)}

\subsection{ECMAScript (aka JavaScript JS)}

\subsection{Dynamic HTML (DHTML)}

\subsection{Server side scripts}

\subsection{Asynchronous JavaScript and XML (AJAX)}

\subsection{Server communication}

\subsection{Graphics -- HTML images vs HTML5 canvas vs Scalable Vector Graphics (SVG)}
To produce a webapp, often a custom user interface and output display is needed.
HTML in combination with CSS and static images offer only very limited abilities, to design a custom dynamic screen elements.
Scalable Vector graphics (SVG) and HTML5 canvas offer more powerful techniques to create dynamic screen content.

\subsubsection{HTML images}
All HTML DOM elements are rectangles and can be placed anywhere on the screen using CSS.
This includes the image tag \T{<img>}, that allows a set of images to be composed to an interface.
JS can be used to change the position of DOM elements, or change the image file they display.
But HTML4 offers no abilities to change the contents of a particualr image, meaning you have to create all the images that possibly could be displayed.
This is fine for simple setup like a menu, but for an advanced user interface, this is not practical.

\subsubsection{HTML5 canvas}
The HTML5 specification defines a new tag, called Canvas \T{<canvas>}.
It allows pixel based image manipulation, by offering a canvas that consists of a discrete, rasterized 2D array of pixels.
By using JS, each pixel of this image grid can be accessed and it's color can be changed dynamically.
There are helper functions that allow the drawing of lines, shapes, paths and images.

Canvas offers fast, low level pixel wise image manipulation.
It is ideal for image manipulation on a per pixel basis like blending.

It offers no scene graph, that keeps track of which structures and shapes were used to draw on the canvas.
If one element changes its position, the canvas needs to be deleted and all objects need to be painted again.
The canvas offers JS callbacks for user actions like click, returning the coordinates of the pixel that was clicked on.
The programmer has to implement a scene graph himself if he wants to know what object / element was clicked on.

\subsubsection{Scalable Vector Graphics (SVG)}
Scalable Vector Graphics (SVG) is a vector image format standard, using a markup language (similar to HTML, based on XML).
SVG defines a continuous coordinate system, on which mathematical constructs are defined on.
This mathematical constructs are objects like paths, basic shapes, test, raster graphics, that are directly stored as a SVG tag in the SVG file.
Since SVG is an XML language, it is comparable with HTML.
CSS can be used to define styles for the SVG objects and JS can be used to manipulate properties.

SVG files offer high level access to graphics, based on shapes.
The direct access to shapes with JS makes it easy to dynamically modify elements or add new shapes.
The SVG DOM can be directly included in the HTML DOM, offering unified access from JS to all HTML and SVG objects.
JS events get fired like in regular HTML.
For example each shape fires an \T{onClick} event if clicked on.


By design, SVG is slower as canvas, because it first needs to rasterized by the browser in order to be drawn on the screen.
Additionally, there was put a lot of effort in Canvas optimization and hardware acceleration lately, whereas SVG hardware support is only minimal.
Due to the lack of hardware optimization, pixel based operations like filters operate slowly.






